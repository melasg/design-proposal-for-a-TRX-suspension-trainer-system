\chapter{Introduction}
The TRX System is a new system of labile suspension weight training purported to develop the user's strength, balance, flexibility, and core stability simultaneously. The TRX system requires the use and design of truss sections, which are held up by legs. There are also corner braces since the elevation of the truss frames requires this for the specific use of leveraging gravity. The requirements of the anchors and frames prevents the entire system from "walking" or rocking while in use that could lead to unsafe physicl use. In this way, the TRX frame is effectively a good way to theoretically balance the loads which requires the foresight, design, and analysis so that failure to balance the loads resulting in toppling or injury does not happen.
\section{Scope}
Despite the widespread use in commercial gyms, homes, parks, the interest in the development of a free-standing structure that can be packaged and marketed to the home user without the exorbitant price of materials for shipping as well as the recommendation for professional installation is great in demand. The demand for a free-standing structure provided for a TRX system anchoring point stabilized with a water filled tank serving as counter weights to the plane trusses is the design consideration for the user going forward. There will be three concepts designed and analyzed and overall the safety of the structure in a free-standing space.\\
Therefore, the proposed design of the structure shall be comprised of two plane trusses separated by 36 inches. The truss shall fit within a specified plane envelope and will be weighed by the anchor points with water counterweights in a plastic tank.\\ 
\section{Objective}
The objective of this design project is to design a planar truss to be used in this proposed concept for anchoring the TRX Suspension System. The volume of water required for stability and the corresponding dimensions for the water tank are also to be determined to verify the feasibility of the design concept. The truss sections will be explored by method of joints as separate design concepts. \\

\chapter{Problem Statement}
\section{Problem Definition}
The problem in the development of a free-standing structure that can be packaged and installed by home TRX users, therefore we need to provide a TRX system anchoring point stabilized at the bottom with a water-filled tank that will serve as a counterbalance. This way, the counterbalance will weigh the two plane trusses at the connection to the ground separated by $36$ inches max. The entire truss sections will fit in a two rectangular mesh truss envelope.\\
\subsection{Engineering requirements}
Structure is designed to support the tension (T) in the strap attached at the anchoring point.\\
When $ \theta = 45$ degrees, the maximum strap tension is $300 lb$.\
When $ \theta = 0$ degrees, the maximum strap tension is $400 lb$.\
Assuming tension is equally distributed to each side of the planar symmetrical trusses:\
Tension is $T = 150 lb$ for $ \theta = 45$ degrees.\\
Tension is $T = 200 lb$ for $ \theta = 0$ degrees.\\
The volume of water required for the stability of the two anchor points and the corresponding dimensions for the water tank are to be determined. \\
The truss members will be tubular aluminum with a maximum length $48$ inches, have load ratings of $2800 lb$ in Tension, and $700 lb$ in Compression. The factor of safety for these values is to be determined.\\
The overall dimensions of the truss will be within the specified truss envelope as drawn in the figure.\\
The two planar trusses are attached with cross braces and the bottom braces to the anchoring points will support the plastic water tank that anchors the entire structure.\\
The total length of the truss members are required, the volume of the water tank are to be determined in a minimized fashion. The water weight is equally distributed and applied to the two joints at ground level.\\
The truss members will snap into specially designed joints that can be modeled as truss pin joints. The design of these joints is not within the specification.\\
\chapter{Solution Procedure}
Calculate the volume of water (tank dimensions) needed to provide the necessary reactions at the ground supports.
Assume the weight of the water is equally distributed and applied to the two joints at ground level.
\section{Design and Analysis Process}
The method of determining forces in the members of the truss is method of joints, where we look at the equilibrium of the pin at the joints. Since the forces are concurrent at the pin, there is no moment equation and only two equations for equilibrium.
$$ \sum F_{x} = 0, \sum F_{y} = 0 $$
To start, we analyze at a point where one known load and at most two unknown forces are there. The weight of each member is divided into two halves and htat is supported by each pin. To an extent, this method is directly inspired by the design and consequently influences the analysis phase. 
\chapter{Conceptual Designs}
Generate three (3) plane truss design concepts that meet the geometric constraints of the problems.
Each design concept shall be statically determinant and stable.
Sketches of the conceptual designs shall include the lengths of the truss members.
\section{Concept 1}
\includegraphics{concept1.png}
\section{Concept 2}
\includegraphics{concept2.png}
\section{Concept 3}
\includegraphics{concept3.png}

\chapter{Analysis of the Proposed Designs}
\section{Results}
Table for each concept listing the length and the calculated force in each members + the total length of the truss members required for each concept.

\subsection{Concept 1: Load Condition A}
\subsection{Concept 1: Load Condition B}

\subsection{Concept 2: Load Condition A}
\subsection{Concept 2: Load Condition B}

\subsection{Concept 3: Load Condition A}
\subsection{Concept 3: Load Condition B}

\chapter{Conclusions}
\section{Design Concept Selection}
Based on the analysis and the engineering requirements, make a recommendation as to which one of the three preliminary design concepts should be considered as a potential solution to this structural design problem. The truss design that does not exceed the truss member load ratings in any given member and has the least total length should be selected. If none of the three meet the load requirements, identify the design that should be considered as a starting point in future work and evaluate the feasibility of the concept. Do you think the concept can be modified to meet the design requirements?
\subsection{Feasibility of the Design Concept}
\chapter{Recommendations}
As the demand for a suspension trainer system that leverages the users' weight and body against the gravity to develop core stability increased after the invention of labile suspension anchoring systems for physical development, so did the methods of creating portable, home systems that were created by home enthusiasts. I would recommend the concept of 