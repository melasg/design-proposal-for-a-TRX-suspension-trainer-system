\chapter{Introduction}

Blended-wing-body(BWB) aircraft configurations becoming more and more attractive due to financial and environmental factors. Besides of high lift to drag ratio that can improve fuel efficiency, there are many other advantages of BWB over conventional 'tube with wings' configuration aircraft, such as lower noisy, huge volumetric capacity, flexible cabin layout potential, significantly DOC reduction, etc. The drawback of BWB is that their structures are not well understood and still many possibilities to find out an optimal layout, while the layout of conventional configurations has been investigated extensively and well understood. Already there are some successfully BWB configurations object design in aircraft industry, like Boeing X-48, Airbus A380, etc. Which potentially reduced fuel consuming by 20\% around. \\
The application of optimization in exploring the structure layout of BWB aircraft has predominately focused on the established methods of parameterizing the geometry and performing size and shape optimization. Different with size and shape optimization, which focusing on geometry parameter or shape of object, topology optimization concentrated on the material distribution of structures. Thus the advantage over other optimization method is that no predefined structure was needed in advance.\\
This paper implements topology optimization method on a BWB configuration unmanned aerial vehicle(UAV) in order to testify the feasible of optimization algorithm as well as the optimal structure that can improve the performances of UAV. The market of UAV is increasing significantly in recent years, probably because of the use of UAV has changed from military to civil. And the endurance of UAV is one of its key performances. Which means the successfully applied of topology optimization on our project is important.



\section{Scope}

The project goal is to reduce the weight of a BWB-integrated-UAV(BITU), which will be introduced in next chapter. Under specific loading and boundary conditions that calculated through X-Foil and AVL, and implemented topology optimization algorithm through commercial software ABAQUS both in 2D and 3D model.

\section{Objectives}

The objective of this project is to minimize compliance(strain energy in this problem) of UAV wing, while maintain a volume fraction of initial design space less than 30\%.


\chapter{Problem Definition}

In this section, several related concepts will be explained for better understanding the process of project.

% \section{BWB-Integrated-UAV(BITU)}
% \includegraphics{1_1}\\[1.6cm]

% \section{Original Structural Configuration of BITU}

\chapter{Solution Procedure}

\chapter{Conceptual Designs}

\chapter{Analysis of the Proposed Designs}
\chapter{Concept Selection}

\chapter{Conclusion}

\chapter{Recommendations}

\chapter{Works Cited}
% [1] New topology optimization method for wing leading edge ribs, DOI 10.2514/1.DC031362

% [2] Vinayak Kulkarni, Anil Jadhav, P.Basker, Finite Element Analysis and Topology Optimization of Lower Arm of Double Wishbone Suspensionusing RADIOSS and Optistruct,International Journal of Science and Research(IJSR), ISSN:2319-7064

% [3] Po Wu, Qihua Ma, Yiping Luo, Chao Tao, Topology Optimization Design of Automotive Engine Bracket, Energy and Power Engineering, 2016,8,230-235

% [4] Qi Wang, Zhenzhou Lu, Changcong Zhou, New Topology Optimization Method for Wing Leading-Edge Ribs, Journal of Aircraft, Vol.48, No.5, September-October 2011

% [5] David Walker, David Liu, Alan Jennings, Topology Optimization of an Aircraft Wing, AIAA SciTech, 5-9 January 2015, Kissimmee, Florida

% [6] Chen Zeng, Rosa Abnous, Souma Chowdhury, Aerodynamic Modeling and Optimization of a Blended-Wing-Body Transitioning UAV, The AIAA Aviation and Aeronautics Forum and Exposition, 5 - 9 June 2017, Denver, Colorado









